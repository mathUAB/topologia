\documentclass{article}
\usepackage[utf8]{inputenc}
\usepackage{amsthm, amsmath, mathtools, amssymb}
\usepackage[left=2.5cm,right=2.5cm,top=3cm,bottom=3cm]{geometry}
\usepackage{titlesec}
\usepackage{color}
\titleformat{\section}{\normalfont\fontsize{11}{15}\bfseries}{\thesection}{1em}{}

\title{\bfseries\Large Seminari 3. Un exemple de compactificació: la d'Alexandroff\\ Correcció}
\author{}
\date{\parbox{\linewidth}{\centering
  Topologia\endgraf
  Grau en Matemàtiques\endgraf
  Universitat Autònoma de Barcelona\endgraf
  Gener de 2022}}

\setlength{\parindent}{0pt}
\begin{document}

\maketitle
En general, aquest informe l'hem trobat molt ben escrit sobretot per la concisió i completesa que ofereix. No obstant això, hem trobat alguns detalls que podrien ser millorables. Els anirem comentant tot seguit, apartat per apartat.

En la introducció hi ha un error tipogràfic: s'escriu una $i$ (que no es defineix enlloc) on hauria de ser una $\iota$. També, al llarg del document hi ha algunes ``X" que haurien de ser ``$X$".

\section{Un petit escalfament}

Els exemples posats són correctes i interessants ja que es fa veure que hi ha immersions que acaben en conjunts de mides diferents, posant 1, 2 o més punts en el conjunt ``d'arribada" respecte el conjunt de ``sortida" de les immersions. Cal, però, fer un petit apunt: en els tres exemples inicials no s'ha especificat si hi ha alguna immersió que sigui, a més, compactificació, que és una de les coses que es demanava. Per tant, l'estudi fet en aquest apartat és adequat, però incomplet. Vegem, doncs, que tant la primera com la segona immersions són compactificacions, però la tercera no ho és.
\begin{enumerate}
  \item Per a la primera, $\iota:\text{id}:(0,1)\hookrightarrow[0,1]$, tenim que $\iota((0,1))=(0,1)$, que és un conjunt dens a $[0,1]$ perquè tot obert $U$ de $[0,1]$ contenint un dels dos extrems, conté també punts al voltant de 0 o 1 al estar treballant en la seva topologia subespai sobre $\mathbb{R}$. Per tant, $U\cap (0,1)\ne\varnothing$.
  \item Per a la segona aplicació, $\iota:(0,1)\hookrightarrow [0,1]/\{0,1\}\cong S^1$ definida per $\iota(t)=(\cos(2\pi t),\sin(2\pi t))$, tenim que si $\overline{0}$ no és obert a $[0,1]/\{0,1\}$, aleshores $\iota$ és una compactificació. Però $\overline{0}$ no és obert a $[0,1]/\{0,1\}$ ja que si ho fos, $\pi^{-1}(\overline{0})=\{0,1\}$ seria obert a $[0,1]$, on $\pi:[0,1]\rightarrow[0,1]/\{0,1\}$ és la projecció al quocient. Per arguments similars als fets en el cas anterior es veu que $\{0,1\}$ no és obert a $[0,1]$.
  \item Finalment, l'última aplicació, $\iota:(0,1)\hookrightarrow{[0,1]^n}$ definida com $(t,a,a,a,\ldots)$ amb $a\in[0,1]$, no és un compactificació $\forall n\geq 2$ ja que $\iota((0,1))=(0,1)\times\prod_{i=1}^{n-1}\{a\}$ no és dens a ${[0,1]}^n$. Per exemple si $a\in(0,1)$, el conjunt $[0,1]\times\overset{(n-1)}{\cdots}\times[0,1]\times(a,1)$ és un obert que no talla $\iota((0,1))$. Si $a=0$ o $a=1$ aleshores podem prendre l'obert $[0,1]\times\overset{(n-1)}{\cdots}\times[0,1]\times(0,1)$ en ambdós casos, que no talla $\iota((0,1))$
\end{enumerate}

\section{La compactificació d'Alexandroff}

En aquest apartat es comença comprovant que el conjunt $\tau^*$ defineix una topologia en $X^*$. La comprovació feta es correcta, encara que hi hagi un petit error tipogràfic: en el moment de veure que la unió arbitrària d'oberts és també un obert, en comptes de dir:
``$\bigcup_{i\in I} U_i \subset \tau^*$ perquè $\bigcup_{i\in I} U_i \subset \tau$ per ser topologia.", s'hauria d'haver dit ``$\bigcup_{i\in I} U_i \in \tau^*$ perquè $\bigcup_{i\in I} U_i \in \tau$ per ser topologia.", al ser els oberts elements del conjunt, i no subconjunts d'aquest.

\par
Després es continua amb una proposició sobre tres propietats generals que verifica la compactificació $X^*$. Les demostracions del primer i tercer punt són perfectament correctes. No obstant això, en la demostració del segon punt, on es comprova que la inclusió de $X$ en $X^*$ defineixi una immersió topològica, ja que no es comprova en cap moment que l'aplicació $\iota$ sigui un homeomorfisme. Això es podria veure, per exemple, verificant que $\iota$ és oberta i/o tancada. Una manera de veure-ho seria la següent:

Sigui $U\subset X$ un obert arbitrari. Aleshores $\iota(U)=U\subset X^*$, que pertany a $\tau^*$ perquè $U\in\tau$. Per tant, $\forall U\in\tau$, $\iota(U)\in\tau^*$. És a dir, $\iota$ és oberta.

\par
Després d'haver demostrat els tres punts de la proposició, s'assegura que $(0,1)^* \cong[0,1]/\{0,1\}$. Però hagués estat interessant explicar, encara que fos breument, com es pot assegurar aquest homeomorfisme. Vegem un exemple de demostració. Considerem l'aplicació $f:{(0,1)}^*\rightarrow[0,1]/\{0,1\}$ definida per $$f=\left\{
  \begin{array}{ccc}
    \overline{x} & \text{si} & x\in(0,1) \\
    \overline{0} & \text{si} & x=\infty
  \end{array}
  \right.$$
Clarament $f$ està ben definida i defineix una bijecció entre els dos conjunts esmentats. L'aplicació $f$ també és contínua. En efecte, si $U=\bigcup_{\overline{x}\in U}\{\overline{x}\}\subset [0,1]/\{0,1\}$ és un obert tenim que: $$f^{-1}(U)=\left\{
  \begin{array}{ccc}
    \bigcup_{\overline{x}\in U}\{x\}                                        & \text{si} & \overline{0}\notin U \\
    \{\infty\}\cup\bigcup_{\overline{x}\in U\setminus\{\overline{0}\}}\{x\} & \text{si} & \overline{0}\in U
  \end{array}
  \right.$$
Com que els oberts a la topologia quocient són els oberts fent antiimatge de la projecció $\pi$, tenim que en el primer cas ($\overline{0}\notin U $) $f^{-1}(U)\in\tau$ i per tant $f^{-1}(U)\in\tau^*$. Si $\overline{0}\in U$, observem que $\pi^{-1}(U)=\{0,1\}\cup\bigcup_{\overline{x}\in U\setminus\{\overline{0}\}}\{x\}$, que és obert perquè $U$ ho és. Així doncs, $\pi^{-1}(U)\cap(0,1)$ també és obert a $(0,1)$. Però aquest últim conjunt coincideix amb $f^{-1}(U)\setminus\{\infty\}$. Per tant, $(0,1)^*\setminus f^{-1}(U)=(0,1)\setminus (f^{-1}(U)\setminus\{\infty\})$ és tancat i compacte. Per tant, $f^{-1}(U)\in\tau^*$.
Finalment, com que ${(0,1)}^*$ és compacte i $[0,1]/\{0,1\}\cong S^1$ és Hausdorff, $f$ defineix un homeomorfisme entre ${(0,1)}^*$ i $[0,1]/\{0,1\}\cong S^1$.

\par
A continuació es dona una demostració molt adequada sobre quina és la condició necessària i suficient perquè la compactificació de $X$ sigui també un espai topològic Hausdorff: cal que $X$ sigui Hausdorff i que cada punt tingui un entorn compacte.
\section{Espais topològics localment compactes}

En aquest apartat es comença veient tres exemples sobre la definició de compacitat local. En el segon exemple s'explica que, per l'observació feta sobre el conjunt $\mathbb{Q}$ en l'apartat anterior, aquest clarament no pot ser un conjunt localment compacte. En el tercer exemple es comenta que el conjunt de Cantor tampoc és localment compacte per un argument similar al de $\mathbb{Q}$. Vegem-ho amb més detall:

Per la definició del conjunt de Cantor no hi pot haver cap interval contingut en aquest conjunt. Per tant, l'interior del conjunt de Cantor sobre $\mathbb{R}$ és buit, d'on es dedueix que l'interior de qualsevol subconjunt compacte té interior buit i, per tant, cap element del conjunt de Cantor es troba en un entorn compacte. Com en el cas de $\mathbb{Q}$, deduïm llavors que el conjunt de Cantor no és localment compacte.
\par
La resta del apartat és correcte.
\section{Una petita proposició de regal}
En aquest apartat tot és correcte.
\end{document}
